% !TeX root = ../main.tex

\chapter{主要研究方法及技术路线}

\section{现有工具漏洞检测能力}

在开始进行实验之前,我们需要对现有工具的检测能力进行充分地调研。只有在了解现有工具的检测能力、检测特点的情况下,我们才能开始进行漏洞标准库的构建和新工具的研发工作。为此,我们调研了当下对Solidity的研究工作,有的工作来自于商业团队,有的来自于学术团队;有的工具已经开源,并具备一定的社区影响力,有的工具发表于计算机顶级国际会议,带来了巨大的科研价值。发表于国际会议的工作,有的没有开源,对于这些没有开源的工作,虽然有论文做详细的说明,但由于不能获取到源代码,我们没办法对系统的内核做更深一步的分析,所以这些工具尽管有一定的学术影响力,我们也只能放弃。对于已经开源的工作,有些工具的开发逻辑不够严谨,或者相关文档不够完备,这些工具我们虽然能取得它们的源代码,但由于无法清晰地分析系统实现,故这些工具我们也无法很好地去分析他们的检测原理和检测能力。综上,在经过我们的筛选后,我们对如下工具在主要漏洞上的检测能力做出了总结,并和我们的系统\textsc{Athena}做比较列于表\ref{tab:detection_capability}。

\begin{table}
  \centering\small
  \caption{现有工具对于主要漏洞的检测能力总结}
  \begin{tabular}{cccccc}
    \toprule
    % after \\: \hline or \cline{col1-col2} \cline{col3-col4} ...
     & \textsc{Slither} & \textsc{Oyente} & \textsc{Smartcheck} & \textsc{Securify} & \textsc{Athena} \\
     \midrule
    可重入漏洞 & $\checkmark$ & $\checkmark$ & $\times$ & $\checkmark$ & $\checkmark$ \\
    意外异常漏洞 & $\checkmark$ & $\times$ & $\checkmark$ & $\times$ & $\checkmark$ \\
    低级调用漏洞 & $\checkmark$ & $\times$ & $\checkmark$ & $\times$ & $\checkmark$ \\
    自毁漏洞 & $\checkmark$ & $\times$ & $\times$ & $\times$ & $\checkmark$ \\
    \bottomrule
  \end{tabular}
  \label{tab:detection_capability}
\end{table}

上表所列的工具皆为静态分析工具,其中\textsc{Oyente}主要使用符号执行分析技术;\textsc{Securify}主要把代码转换成Datalog语言,并使用\textsc{Souffle}进行分析。\textsc{Slither}和\textsc{Smartcheck}采用的是传统的静态分析技术,即通过分析源代码得到程序的控制流图,并在控制流图上用预先设定好的匹配规则去寻找漏洞。从表中不难看出,使用传统静态分析技术的工具分析能力都比较不错,\textsc{Slither}支持我们提及的所有漏洞的检测,\textsc{Smartcheck}不支持两个漏洞的检测;而使用符号执行分析技术,包括使用其他静态分析技术的工具,对主要漏洞的支持都不太好,甚至只支持一个漏洞的检测。值得一提的是,这两个工具\textsc{Oyente}和\textsc{Securify}皆是在源代码编译之后产生的字节码上进行软件分析的,字节码的分析提供的信息更少,相比之下\textsc{Slither}和\textsc{Smartcheck}都是对源代码或者中间语言进行分析,故我们推测是由于技术路线的差异造成它们在不同漏洞上的分析难度不同,也就没办法支持所有漏洞的分析任务。

