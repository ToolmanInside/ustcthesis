% !TeX root = ../main.tex

\begin{abstract}
区块链,包括以太坊、智能合约等等,从方方面面改变着人们的生活。这项技术给各种行业带来了革新,如分布式银行,分布式能源管理系统等等。它给我们带来了巨大的技术便利,同时也可能带来巨大的经济损失。在2016年,由黑客发起的针对DAO合约的攻击共造成了数百万美元的损失。一方面,区块链技术带来的技术革命激励着人们不断创新;另一方面,区块链应用,作为典型的软件,理应接受软件分析技术的检验。

软件分析技术,包括软件静态分析技术、动态分析技术、形式验证技术等等,有着多种不同的形式,各种分析技术有不同的特点,在处理不同类型的问题时,应该采用不同的分析技术。在这篇工作中,我们使用静态分析技术,辅以克隆分析技术,来构建漏洞检测系统。目前在智能合约软件上的分析工作并不多,智能合约软件目前还属于一个新领域。这篇工作将改善现有系统的检测效果,调研字节码的相关工作,并提出标准漏洞库,这几项目标将改善现有智能合约领域缺乏的开发生态,并为之后的工作打下坚实的基础。

我们调研了著名的4个智能合约的漏洞,对它们的基本形式,主要特点,危害性都在文中做出了描述。我们也调研了现有工具在这几个主要漏洞上的检测能力,并对工具的检测能力的原因做了简单分析,我们发现,极少有工具能覆盖所有类型的漏洞。除此之外,我们调研了以太坊虚拟机对生成字节码的影响,探寻了字节码背后的奥秘。

为了有效改善现有的软件分析技术,在这篇工作中,我们提出结合传统静态分析技术与克隆分析技术。通过静态分析技术我们归纳总结了四种漏洞签名,再根据克隆分析技术去寻找漏洞,我们还在后面加入了九种安全盾技术以降低系统的误报数量。在整个系统中,由于标准漏洞库的缺乏,我们在部分环节加入了领域专家的审计以保证我们提出的漏洞库的公正性。最后,我们提出了基于软件分析的漏洞检测系统,Athena。

经过我们大量的实验,我们明显地看出,Athena在这四种漏洞上有着不俗的表现,整体准确率和整体覆盖率都是最高的。至于系统的运行效率,Athena的运行速度和最快的工具仍有差距,但也在可接受的范围内。我们认为,我们提出的Athena系统,改进了现有的检测技术,达到了更好的检测效果,同时也拥有着不俗的运行效率。
\keywords{以太坊;智能合约;软件分析技术;克隆检测技术;软件漏洞}
\end{abstract}

\begin{enabstract}


  \enkeywords{PhD}
\end{enabstract}
