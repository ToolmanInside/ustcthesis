% !TeX root = ../main.tex

\chapter{序言}

\section{课题研究意义及国内外现状分析}

\subsection{研究意义}
自2009年“中本聪”发布了比特币的第一篇论文以来,区块链技术受到了越来越多的关注。虽然相比传统的计算机技术,区块链技术还比较年轻,但是区块链技术本身带来的新型货币体系,包括其在各行各业的应用:去中心化的新型银行\cite{blockchain-application},基于去中心化智能合约的新型能源管理系统\cite{blockchain-in-energy}等等,无不凸显这一崭新技术在各行各业带来的革命性意义。

相比传统的中心化服务器,区块链技术的核心为去中心化的服务器系统,即网络中的每个运行区块链软件的设备都是一个独立的对等节点\cite{区块链技术及其应用研究}。相比传统的中心化服务器系统,区块链的网络中的所有节点在进行交易时不依赖于中间人,所有人都维护着同一个“账本”,并通过密码学算法对自己的交易信息进行签名加密,最后通过计算量证明来保证账本的有效和可靠性。区块链技术中体现出的去中心化,独立平等和无法随意篡改等等特性吸引了大量学者、企业、研究单位的关注。

在区块链的所有应用中,以太坊(Ethereum)是最火热的应用之一。以太坊是一个基于区块链技术的平台,发起的初衷为“建立无法被停止的应用”\cite{eth-intro}。他使用以太币(Ether)作为平台的主要货币,用密码学的算法来保证交易的正常进行。以太坊使用智能合约作为交易的主要媒介,网络中的各个节点通过智能合约完成交易、互动等功能。从2017年开始,以太坊的交易量开始高速增长\cite{eth-chart},越来越多的用户开始使用智能合约完成以太币的交易,区块链游戏的交互等等任务。

智能合约是由一种图灵完整的编程语言,Solidity,编写而成的。随着使用智能合约的人越来越多,保证智能合约使用安全的呼声也越来越高,很多学者和研究机构开始注意这门新语言,并用软件分析的方法(静态、动态分析等等)去研究这门新语言。2016年9月,一场被称为“DAO攻击”的事件窃取了价值数百万美元的以太币\cite{dao-attack}。保证以太坊软件的安全迫在眉睫。

软件分析方法,包括静态分析、动态分析、形式验证、模糊测试、代码克隆检测等等方法,是保证软件安全、分析软件特性的必经之路。以上提到的几种分析方法,各有特点,应根据使用场景进行考虑,选取合适的分析方法。其中,静态分析主要以分析软件的抽象语法树(AST),控制流图(CFG)等作为主要的分析对象,辅以特征提取\cite{deckard}、图匹配等方法,分析软件的特点;动态分析主要侧重于使用符号执行、约束求解等等方法,分析软件的特点,相比静态分析耗时更长,但准确度更高;形式验证能保证较高的软件安全等级,主要侧重于对软件流程形式化描述的分析;模糊测试的特点在于通过种子(SEED)来制作测试用例生成器,对软件的攻击面(可能存在漏洞或者缺陷的地方)输入测试用例,通过对测试用例的不断变化来达到测试软件漏洞的目的;代码克隆借助LCS(最长公共子串)算法,分析执行迹线,仿真等等方法来分析代码的相似程度。

同时,我们在从Ehterscan爬取了200万个的智能合约之后,分析统计发现,只有不到10\%的智能合约是经过验证(能找到对应源代码)的,其余智能合约以字节码(Bytecode)的形式存在于以太坊的平台上。而传统的分析方法主要以白盒分析为主,面对字节码无能为力。而这部分合约又恰好牵扯了大量的以太坊资金,如果这部分以字节码形式存在的合约的安全不能得到保证,不仅易使以太坊的用户遭受大量的经济损失,扰乱区块链的经济秩序,也会带来更多的社会不安定因素。因此,我们需要一个能直接对字节码进行分析的系统来保证这部分软件的安全。

\subsection{国内外现状分析}



\subsubsection{软件分析方法国内外现状}

目前,国内外已有大量的软件分析方法的工作。静态的分析方法通过抽象语法树(AST),控制流图(CFG),程序依赖图(PDG)等分析程序的特点。动态分析方法比如\cite{survey-dynamic}就总结了目前流行的动态分析方法,动态分析方法具有精确度高的优点,但是覆盖率和耗时长也是在使用动态方法的时候不得不考虑的问题。但更多的工作是把静态分析和动态分析结合起来,既弥补了动态分析在覆盖率上的不足,也部分解决了静态分析准确度不高的问题。
由于智能合约还属于新事物,在智能合约上的软件分析方法还以静态分析和动态分析为主,当然也有使用形式验证、模糊测试等方法分析智能合约的。静态方法包括Slither\cite{slither},是通过分析Solidity编译器生成AST,再将AST转换成IR(中间语言)和CFG(控制流图);动态方法包括Oyente\cite{oyente}、Securify\cite{securify}、Manticore\cite{manticore}都是使用了约束求解的方法分析程序的CFG,然后根据约束求解的结果来分析软件的特点,其中Oyente和Manticore均使用了在软件分析领域小有名气的约束求解器Z3\cite{z3},而Securify则使用的是Souffle\cite{souffle}进行约束求解。同样的,IBM的以太坊研究团队提出了Zeus\cite{zeus},使用了形式验证的方法实现了对Solidity软件的分析。而Echidna\cite{echidna},则是目前智能合约软件唯一的自动化模糊测试工具。智能合约虽然出生不久,但是分析智能合约的软件已经百花齐放。


\paragraph{四级节标题}

\subparagraph{五级节标题}

Lorem ipsum dolor sit amet, consectetur adipiscing elit, sed do eiusmod tempor
incididunt ut labore et dolore magna aliqua.
Ut enim ad minim veniam, quis nostrud exercitation ullamco laboris nisi ut
aliquip ex ea commodo consequat.
Duis aute irure dolor in reprehenderit in voluptate velit esse cillum dolore eu
fugiat nulla pariatur.
Excepteur sint occaecat cupidatat non proident, sunt in culpa qui officia
deserunt mollit anim id est laborum.



\section{脚注}

Lorem ipsum dolor sit amet, consectetur adipiscing elit, sed do eiusmod tempor
incididunt ut labore et dolore magna aliqua.
\footnote{Ut enim ad minim veniam, quis nostrud exercitation ullamco laboris
  nisi ut aliquip ex ea commodo consequat.
  Duis aute irure dolor in reprehenderit in voluptate velit esse cillum dolore
  eu fugiat nulla pariatur.}
