% !TeX root = ../main.tex

\chapter{序言}

\section{课题研究意义及国内外现状分析}

\subsection{研究意义}
自2009年“中本聪”发布了比特币的第一篇论文以来,区块链技术受到了越来越多的关注。虽然相比传统的计算机技术,区块链技术还比较年轻,但是区块链技术本身带来的新型货币体系,包括其在各行各业的应用:去中心化的新型银行\cite{blockchain-application},基于去中心化智能合约的新型能源管理系统\cite{blockchain-in-energy}等等,无不凸显这一崭新技术在各行各业带来的革命性意义。

相比传统的中心化服务器,区块链技术的核心为去中心化的服务器系统,即网络中的每个运行区块链软件的设备都是一个独立的对等节点\cite{区块链技术及其应用研究}。相比传统的中心化服务器系统,区块链的网络中的所有节点在进行交易时不依赖于中间人,所有人都维护着同一个“账本”,并通过密码学算法对自己的交易信息进行签名加密,最后通过计算量证明来保证账本的有效和可靠性。区块链技术中体现出的去中心化,独立平等和无法随意篡改等等特性吸引了大量学者、企业、研究单位的关注。

在区块链的所有应用中,以太坊(Ethereum)是推行加密货币的先驱之一。加密货币是一种经过密码学技术加密的数字货币。和其他货币如美元不同的是,加密货币没有实物,而且他们并不会受到中央银行或者政府部门的管理。加密货币独特的地方在于任何一个人都能使查看过往合约的所有交易记录。以太坊是一个基于区块链技术的平台,发起的初衷为“建立无法被停止的应用”\cite{eth-intro}。以太坊使用了和比特币不同的技术路线,它提供了一个交易平台,在这个平台上用户解决标记(Tokens)去创建和运行各种应用。以太坊使用以太币(Ether)作为平台的主要货币,用密码学的算法来保证交易的正常进行。以太坊使用智能合约作为交易的主要媒介,网络中的各个节点通过智能合约完成交易、互动等功能。从2017年开始,以太坊的交易量开始高速增长\cite{eth-chart},越来越多的用户开始使用智能合约完成以太币的交易,区块链游戏的交互等等任务。

智能合约是由一种图灵完整的编程语言,Solidity,编写而成的。Solidity是一种面向对象、高层次的编程语言。Solidity受到了C++、Python和JavaScript的影响,使用了以太坊虚拟机(Ethereum Virtual Machine)进行编译运行。Solidity是一门静态编程语言,支持继承关系,库调用和完整的用户自定义类型。使用Solidity能够创建投票、募集资金、盲选等等智能合约应用。随着使用智能合约的人越来越多,保证智能合约使用安全的呼声也越来越高,很多学者和研究机构开始注意这门新语言,并用软件分析的方法(静态、动态分析等等)去研究这门新语言。2016年9月,一场被称为“DAO攻击”的事件窃取了价值数百万美元的以太币\cite{dao-attack}。保证以太坊软件的安全迫在眉睫。

软件分析方法,包括静态分析、动态分析、形式验证、克隆分析等等方法,是保证软件安全、分析软件特性的必经之路。以上提到的几种分析方法,各有特点,应根据使用场景进行考虑,选取合适的分析方法。其中,静态分析技术不直接运行软件,主要的分析对象为软件的源代码,分析代码的抽象语法树(AST),控制流图(CFG),辅以特征提取\cite{deckard}、图匹配等方法,分析软件的特点,静态分析技术也使用符号执行技术,将程序控制流图上的变量使用符号进行替代,使用各种约束求解器进行求解;动态分析技术是运行程序或使用虚拟机,并使用大量的测试用例做输入,以期程序产生异常的分析技术,它使用一些软件测试技术,如代码覆盖率(Code Coverage Rate)来保证测试用例能够覆盖到程序的所有运行路径,动态分析相比静态分析耗时更长,但准确度更高;形式验证分析的对象不是软件的实际代码,而是将软件的实现以另一种更形式化的方式描述,最后通过数学计算验证软件的安全是合理的,形式验证分析技术有很强的理论性,并且能保证较高的软件安全等级;代码克隆借助各种相似度计算算法,从漏洞代码入手,寻找在语义上或者程序结构上与漏洞代码相似的代码,常常辅以其他的技术来提高准确度。

\subsection{国内外现状分析}

\subsubsection{智能合约漏洞国内外研究现状}

从2016年开始,越来越多的学者加入对智能合约的研究。出现了很多优秀的整理智能合约上容易出现的漏洞的工作,比如\cite{survey-on-attacks},不仅收集了很有名的可重入漏洞,也有诸如错误乱序这种较难发现的漏洞,文中对各个漏洞也附上了代码示例。这些示例显然不是真实的例子,是作者自己根据漏洞的原理自己构造的,但浅显易懂,为刚开始接触智能合约的研究者提供了详实的资料。也有的团队\cite{survey-on-smart-contracts},从分析以太坊的使用记录入手,根据他们的数据,目前大部分的智能合约都来交易以太币或者管理钱包,极少部分智能合约被用来进行游戏的交互和其他应用,这也说明大部分的智能合约是和经济安全紧密联系起来的。如果智能合约的安全受到侵害,会造成大量的经济损失,这更说明本文们在智能合约的基础上研究软件安全是十分有必要的。

\subsubsection{软件分析方法国内外现状}

目前,国内外已有大量的软件分析方法的工作。静态的分析方法通过抽象语法树(AST),控制流图(CFG),程序依赖图(PDG)等分析程序的特点。动态分析方法比如\cite{survey-dynamic}就总结了目前流行的动态分析方法,动态分析方法具有精确度高的优点,但是覆盖率和耗时长也是在使用动态方法的时候不得不考虑的问题。但更多的工作是把静态分析和动态分析结合起来,既弥补了动态分析在覆盖率上的不足,也部分解决了静态分析准确度不高的问题。

由于智能合约还属于新事物,在智能合约上的软件分析方法还以静态分析和动态分析为主,当然也有使用形式验证、模糊测试等方法分析智能合约的。静态方法包括\textsc{Slither}\cite{slither},是通过分析Solidity编译器生成AST,再将AST转换成IR(中间语言)和CFG(控制流图);动态方法包括\textsc{Oyente}\cite{oyente}、\textsc{Securify}\cite{securify}、\textsc{Manticore}\cite{manticore}都是使用了约束求解的方法分析程序的CFG,然后根据约束求解的结果来分析软件的特点,其中\textsc{Oyente}和\textsc{Manticore}均使用了在软件分析领域小有名气的约束求解器Z3\cite{z3},而\textsc{Securify}则使用的是\textsc{Souffle}\cite{souffle}进行约束求解。同样的,IBM的以太坊研究团队提出了\textsc{Zeus}\cite{zeus},使用了形式验证的方法实现了对Solidity软件的分析。而\textsc{Echidna}\cite{echidna},则是目前智能合约软件唯一的自动化模糊测试工具。智能合约虽然出生不久,但是分析智能合约的软件已经百花齐放。

%\subsubsection{字节码分析方法国内外发展现状}
%
%目前,Solidity语言的编译和执行都是通过EVM(以太坊虚拟机)进行的,相比在X86平台的C编译器生成的二进制文件,字节码是智能合约软件的最底层实现,直接在EVM上执行。想要分析字节码,最开始的一步就是要将字节码反编译过来。但是,彻底的反编译几乎是不可能的,Xiaozhu Meng的工作\cite{binary-not-ez}说明了在逆向工程中可能会遇到的问题,比如变量名无法复原、无法完全复原语义等等,其他的逆向工程的工作比如\textsc{DIVINE}\cite{divine}和\textsc{TIE}\cite{tie}研究了在逆向工程中复原变量名和复原变量原类型的可能性,但效果始终不太令人满意。近几年在智能合约上也有一些工作提供了很有创新的操作,比如\textsc{Erays}\cite{erays},他们提出了将字节码转换成单赋值语句(SSA),在通过特殊的聚合手段加强SSA的语义信息,得到非常接近源代码的中间语言;而\textsc{Securify}\cite{securify},则直接将字节码分析为控制流图,再在控制流图上分析操作的执行序列来寻找软件的漏洞。在智能合约领域已经出现了很多不错的分析字节码的工具,但不可否认的是,他们都有着明显的缺陷,比如\textsc{Erays}没有办法对复杂的语义进行整合,而\textsc{Securify}在大规模数据集上的表现很糟,可扩展性较差。为了解决这样的问题,本文提出用软件克隆分析的方法来分析智能合约代码。

\subsubsection{软件克隆分析方法国内外发展现状}\label{sec:clone_intro}
提到软件克隆分析就不得不提到ChanChal等人的工作\cite{survey-on-clone},这篇文章分析了软件克隆分析法的优势和局限性,指出了克隆的四种级别,也推荐了一些克隆的分析方法。虽然这篇工作推荐的分析方法有部分已经落后于时代,但其中对克隆级别和克隆关键问题的讨论令人印象深刻。大体来说,各家对于什么是克隆、克隆的准确定义争论已久。同时,文中也支出了克隆的四种级别,第一类克隆、第二类克隆、第三类克隆和第四类克隆。其中,第一类克隆也叫完全克隆,两段代码只有空格和注释的不同。第二类克隆包括重命名克隆,即两段代码对变量的命名有差异。第三类克隆包括代码执行顺序,代码间隔的不同。而第四类克隆是语义上的相似,两段代码可以结构完全不同,但是功能是完全相同的。在这四种克隆类型中,第一类到第二类克隆因为比较简单,可以借助字符串匹配完美的实现,第三类克隆需要借助一些克隆分析手段,如最长公共子串(LCS)来实现,而第四类克隆需要加入更多的语义分析才能实现,是目前克隆的难点之一。

在针对二进制程序的克隆检测中,有工作比如\textsc{DECKARD}\cite{deckard},通过分析编译器生成的AST,从AST中的关键节点提取关键信息,并将提取的信息扁平化,再比较特征向量在高维空间中的距离来分析两个软件代码的相似度;有工作\textsc{BinGo}\cite{bingo}采用了多样的分析方法,通过仿真,函数的执行路径,控制流图等方式来分析两个软件代码的相似度。

软件克隆有不得不面临的困难,可以也有得天独厚的优势。软件克隆方法在保证了一定的准确性的前提下,有着较好的可扩展性,在面对大量数据集的时候也能保证一定的的分析速度。本文的系统将把静态分析同克隆的方法结合起来,实现在字节码上的的软件分析。

\section{选题的创新性和先进性}

本系统是针对智能合约的字节码开发的软件分析系统。目前,世界上还没有一个完备的系统能够用软件克隆的方法实现对智能合约的软件分析工作;而智能合约软件内存在着严重的代码克隆现象,漏洞随着代码克隆进行广泛传播。使用克隆分析技术,能有效改善当前工具的检测效率。因此,这不仅是创新的工作,也是具有有很大贡献的工作。同时,这些智能合约软件和以太坊的经济安全息息相关,保证这部分智能合约的软件安全是很有价值的工作。
%同时,在以太坊上有大量的智能合约是以字节码的形式存在,这些合约和以太坊的经济安全息息相关,保证这部分智能合约的软件安全是很有价值的工作。

\section{主要研究内容}

该选题技术难度偏难,主要的难点在于以下几点:如何总结现有的漏洞主要攻击模式、如何构建标准漏洞库、如何将克隆的方法同静态分析的方法结合起来,选题能满足研究生论文的研究要求;工作量主要体现在漏洞的主要攻击模式上,需要阅读大量的漏洞代码,并且克隆方法的开发也需要大量的工作,但能够在预定的时间内完成课题研究的内容。针对以上难点,本文提出以下主要研究内容。

\subsection{调研主要的几种Solidity漏洞}

智能合约和以太坊的经济安全息息相关,因此有不少不法分子妄图借助智能合约的软件漏洞,窃取钱财。为了更好的完成软件分析技术的开发工作,首先要做的就是充分调研目前现有的智能合约漏洞,这样不仅能更清楚地分析出黑客利用漏洞的攻击模式,也能细微地调整软件分析技术在不同漏洞上的表现。

\subsection{调研现有智能合约分析工具}

随着以太坊和智能合约的逐渐火热,有越来越多的团队加入智能合约的分析工作之中,有的团队提出分析技术,而更多的团队则是在原来的分析方法上做出针对智能合约改进。其中,有的团队使用了静态分析技术,如\textsc{Slither}\cite{slither}、\textsc{SmartCheck}\cite{smartcheck}和\textsc{Securify}\cite{securify}等等;有的团队使用了动态分析技术,如\textsc{Oyente}\cite{oyente}、\textsc{Mythril}\cite{mythril}、\textsc{MythX}\cite{mythx}、\textsc{Manticore}\cite{manticore}等等;有的团队使用了形式验证的技术,如\textsc{Zeus}\cite{zeus};也有的团队使用了模糊测试的技术,如\textsc{Echidna}\cite{echidna}等等。为了提出一个经得起考验的标准数据集,广泛调研目前各家的技术是非常有必要的。

%\subsection{调研现有智能合约字节码反编译技术}
%
%反编译,特别是字节码的反编译,并不容易。首先,智能合约还尚属新事物,各类工具还不齐全,分析能力很有限;其次,以太坊虚拟机的特性也给反汇编技术的开发工作带来了困难。为了更好地开发出字节码反编译技术,需要对字节码的产生过程、字节码的特点、字节码的主要结构、操作码到字节码的映射关系等等有充分的了解。

\subsection{软件克隆分析技术在漏洞检测上的应用}

软件克隆分析技术并不是第一次被使用于漏洞检测系统上,通过调研,本文发现之前的工作如VUDDY\cite{vuddy}等使用基于克隆分析技术的漏洞检测系统在大规模数据集上取得了良好的检测效果。但目前还未有在智能合约,Solidity软件上使用克隆分析技术进行漏洞检测的工作。本文需要充分调用现有的和之前的相关工作,克服智能合约软件平台特性带来的困难,设计一套漏洞检测系统。

\section{拟解决的关键问题}

\subsection{如何保证标准漏洞库的标准性和代表性}

智能合约发展至今,还没有一个团队或者企业提出这样的一个标准数据集。本文从互联网上搜集合约代码数据,根据之前出现过的合约漏洞进行人工统计和分析。但即使是人工查验,也有误判和漏判的情况出现,这样的标准数据集是否足够标准将会成为一个值得讨论的问题。

\subsection{现有漏洞检测工具的真实表现如何}

智能合约上的漏洞检测工具经过近几年的发展,已经可以使用多种漏洞检测方法对漏洞进行定位、检测。这些工具有的选择投稿会议,有的选择进行开源以扩大社区影响力。这些工具各有优缺点,在不同检测方法上各展身手,但是本文仍需要一个囊括主流检测工具的横向比较,只有这样,才能对工具的真实性能做出准确评估,更进一步,这对比较各检测方法在智能合约软件上的适用性也有很大的帮助。

%智能合约上的字节码反编译技术是本文研究的重点。如何通过反编译技术捕捉语义,可借鉴于之前在Java和Android上的一些工作。即便这样,本文也不得不面临着EVM字节码自带的难题:相比Java的字节码,EVM的字节码提供的信息更少。反编译技术的研发成功与否,直接影响了本文下一步将克隆技术应用与字节码的可行性。

\subsection{使用克隆技术进行漏洞检测的效果是否符合预期}

在完成调研工作之后,本文需要实现将软件克隆技术应用于漏洞检测系统。以往的软件克隆分析技术,包括提取操作序列、提取函数特征等等,在智能合约的这门新语言上是否有效,要打一个问号。如果单一的技术行不通,那可以考虑加入其他静态分析技术或者动态分析技术,不断调整和优化软件分析系统。