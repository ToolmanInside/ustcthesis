\chapter{总结}

本文总共五个章节,在第一章中,我们介绍了以太坊,智能合约以及我们工作选题的先进性、创新性,和我们工作主要的研究内容及拟解决的关键问题;在第二章中,我们介绍了智能合约的相关背景知识,包括智能合约的常见漏洞和前沿的智能合约软件分析工具,除此之外,我们还简单介绍了Solidity字节码的相关知识;在第三章中,我们引入了我们主要的研究方法及技术路线,我们首先通过调研现有工具的检测能力和之前软件克隆分析的相关文献,确定了克隆分析技术在智能合约软件漏洞检测上的可行性,接下来,我们提出基于漏洞签名的漏洞检测技术和提高检测准确率的CPT技术;在第四章中,我们配置了相关实验,并针对三个研究问题,我们系统是否准确,我们系统覆盖率如何,我们系统的运行效率如何做出了回答。

在第一章中,我们列举了国外智能合约研究、软件分析研究方面的进展,进而提出我们选题的先进性和创新性。接下来我们描述了我们的工作的主要研究内容,内容涉及漏洞、前沿工具、字节码以及克隆分析技术。最后,我们针对这几项研究内容提出了我们这篇工作要解决的几个关键问题。

在第二章中,我们主要介绍了本课题需要了解的相关背景知识。首先,我们针对智能合约中常见的四种漏洞做了简单介绍并举例;其次,我们对现有的分析工具进行了分类并介绍特点;最后,我们还介绍了智能合约字节码的相关知识。

在第三章中,我们首先介绍了现有工具的漏洞检测能力,并调研了之前的克隆分析技术文献作为我们的方法论指导。然后,我们观察智能合约代码的特点,分析和讨论了在智能合约软件上使用克隆分析技术的可行性。在这之后,我们提出了基于规则的漏洞检测技术,我们观察现有工具运行原理并加入自己的理解,成功提取了各类型的漏洞签名。在经过大量阅读智能合约代码之后,我们也提出使用CPT技术来改善检测系统的准确率。针对各种不同的漏洞,我们共提出了九种CPT技术。最后,我们提出了我们系统Athena的实现算法。

在第四章中,我们进行了实验。我们首先对实验的配置进行讲解,并提出实验要解决的三个研究问题。接下来,我们根据实验结果对这三个问题进行了一一解答,这三个问题包括我们的系统的检测准确率如何,我们系统的检测出的真实漏洞覆盖率如何以及我们系统的运行效率如何。

最后,我们在这里对以上章节做出了总结。我们提出的Athena系统在准确率和覆盖率上对比现有前沿工具都有着不小的优势,但是我们的系统仍有不小的改进空间,如漏洞签名的改进、加入数据依赖分析等等。我们的系统在将来会越来越可靠,也希望能带给智能合约软件的开发者更多的便利。