\chapter{总结}

本文总共五个章节,在第一章中,我介绍了以太坊,智能合约以及工作选题的先进性、创新性,和工作主要的研究内容及拟解决的关键问题;在第二章中,我介绍了智能合约的相关背景知识,包括智能合约的常见漏洞和前沿的智能合约软件分析工具,除此之外,还简单介绍了Solidity字节码的相关知识;在第三章中,我引入了主要的研究方法及技术路线,首先通过调研现有工具的检测能力和之前软件克隆分析的相关文献,确定了克隆分析技术在智能合约软件漏洞检测上的可行性,接下来,提出了基于漏洞签名的漏洞检测技术和提高检测准确率的CPT技术;在第四章中,我配置了相关实验,并针对三个研究问题,系统是否准确,系统覆盖率如何,系统的运行效率如何做出了回答。

在第一章中,我列举了国外智能合约研究、软件分析研究方面的进展,进而提出本文选题的先进性和创新性。接下来本文描述了我工作的主要研究内容,内容涉及漏洞、前沿工具、字节码以及克隆分析技术。最后,我针对这几项研究内容提出了这篇工作要解决的几个关键问题。

在第二章中,我主要介绍了本课题需要了解的相关背景知识。首先,我针对智能合约中常见的四种漏洞做了简单介绍并举例;其次,我对现有的分析工具进行了分类并介绍特点;最后,我还介绍了智能合约字节码的相关知识。

在第三章中,我首先介绍了现有工具的漏洞检测能力,并调研了之前的克隆分析技术文献作为我的方法论指导。然后,我观察了智能合约代码的特点,分析和讨论了在智能合约软件上使用克隆分析技术的可行性。在这之后,我提出了基于规则的漏洞检测技术,并观察现有工具运行原理并加入自己的理解,成功提取了各类型的漏洞签名。在经过大量阅读智能合约代码之后,我也提出使用CPT技术来改善检测系统的准确率。针对各种不同的漏洞,我共提出了九种CPT技术。最后,我提出了漏洞检测系统Athena的实现算法。

在第四章中,我进行了实验。首先对实验的配置进行讲解,并提出实验要解决的三个研究问题。接下来,我根据实验结果对这三个问题进行了一一解答,这三个问题包括Athena系统的检测准确率如何,系统的检测出的真实漏洞覆盖率如何以及系统的运行效率如何。

最后,我在这里对以上章节做出了总结。我提出的Athena系统在准确率和覆盖率上对比现有前沿工具都有着不小的优势,但是本系统仍有不小的改进空间,如漏洞签名的改进、加入数据依赖分析等等。我的系统在将来会越来越可靠,也希望能带给智能合约软件的开发者更多的便利。
